\part*{Contribuer}
\chapter*{Comment Contribuer à ce livre ?}
\addcontentsline{toc}{part}{Contribuer}
\markboth{Contribuer}{Contribuer}
\setcounter{tocdepth}{1}
\setcounter{chapter}{0}

Nous sommes ravis que vous envisagiez de contribuer à "La Quête de l'Expérience"! 

Votre aide et vos idées sont les bienvenues.

Les sources de ce livre sont disponible sur GitHub à l'adresse suivante : 

\href{https://github.com/kdelfour/Ebook_The_path_of_the_young_developer_Your_quest_for_experience}{\url{https://github.com/kdelfour/Ebook_The_path_of_the_young_developer_Your_quest_for_experience}}

\section*{Code de Conduite}

En premier lieu, veuillez lire et respecter notre Code de Conduite. Nous nous attendons à ce que tous les contributeurs respectent ces directives pour assurer une communauté accueillante et respectueuse.

\subsection*{Comment puis-je contribuer ?}

Il y a plusieurs façons de contribuer à ce projet :

\begin{itemize}
    \item \textbf{Corrections d'erreurs ou de fautes d'orthographe} :  Si vous trouvez des erreurs dans le livre, qu'il s'agisse d'erreurs factuelles ou de simples fautes d'orthographe, n'hésitez pas à les corriger.
    \item \textbf{Ajout de nouveaux contenus} : Si vous avez des idées de nouveaux sujets à aborder ou de nouvelles perspectives à ajouter, nous serions ravis de les entendre.
    \item \textbf{Traductions} : Si vous êtes en mesure de traduire le livre dans une autre langue, cela serait une contribution précieuse.
\end{itemize}

\section*{Processus de Contribution}

Pour contribuer, suivez ces étapes :
\begin{itemize}
    \item \textbf{Forkez} le projet.
    \item \textbf{Créez votre nouvelle branche} : `git checkout -b my-new-feature`.
    \item \textbf{Commitez vos changements} : `git commit -am 'Add some feature'`
    \item \textbf{Poussez à la branche} : `git push origin my-new-feature`.
    \item \textbf{Soumettez} une pull request
\end{itemize}
\vspace {0.5cm}

Avant de soumettre votre pull request, assurez-vous que votre contribution respecte les lignes directrices suivantes :

\begin{itemize}
    \item Vérifiez l'orthographe et la grammaire.
    \item Gardez le style de rédaction cohérent avec le reste du livre.
    \item Si vous ajoutez du contenu, assurez-vous qu'il est pertinent et précis.
    \item Si vous faites des modifications majeures, veuillez ouvrir une issue pour en discuter avant de faire la pull request.
\end{itemize}

\section*{Des questions ?}

Si vous avez des questions ou des préoccupations, n'hésitez pas à ouvrir une issue. Nous ferons de notre mieux pour vous répondre rapidement.

Merci pour votre contribution !

