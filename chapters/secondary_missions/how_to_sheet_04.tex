\chapter{Fiche pratique : Rédaction d'un CV de développeur·se}

Rédiger un CV de développeur·se peut être un défi, surtout si vous êtes un·e jeune développeur·se sorti·e d'école, en fin d'alternance ou en reconversion professionnelle. Vous devez montrer aux employeur·se·s potentiel·le·s que vous avez les compétences et l'expérience nécessaires pour le poste, même si vous n'avez pas beaucoup d'expérience professionnelle. Voici un guide pratique pour vous aider à rédiger un CV de développeur·se impressionnant.

\section{Conseils pour la rédaction d'un CV de développeur·se}

Voici quelques conseils pour vous aider à rédiger votre CV :

\begin{itemize}
    \item \textbf{Mettez en avant vos compétences techniques :} Listez les langages de programmation, les frameworks, les outils et les technologies que vous maîtrisez. Assurez-vous d'inclure ceux qui sont pertinents pour le poste auquel vous postulez.

    Les compétences techniques sont essentielles pour un·e développeur·se. Elles sont souvent la première chose que les employeur·se·s recherchent lorsqu'ils·elles examinent un CV. En mettant en avant vos compétences techniques, vous montrez aux employeur·se·s que vous avez les compétences nécessaires pour effectuer le travail.

    \item \textbf{Incluez vos projets personnels :} Si vous avez des projets personnels ou des contributions à l'open source, assurez-vous de les inclure dans votre CV. Cela peut aider à démontrer vos compétences pratiques et votre passion pour le développement.

    Les projets personnels peuvent démontrer vos compétences pratiques et votre passion pour le développement. Ils montrent que vous êtes capable de mener à bien un projet en dehors d'un environnement de travail structuré. C'est particulièrement important si vous n'avez pas beaucoup d'expérience professionnelle.

    \item \textbf{Mentionnez votre formation :} Incluez vos diplômes et certifications, ainsi que toute formation pertinente que vous avez suivie. Si vous avez suivi des cours en ligne ou des bootcamps de programmation, n'hésitez pas à les mentionner.

    Même si la formation formelle n'est pas toujours nécessaire pour devenir développeur·se, elle peut aider à établir votre crédibilité et à montrer que vous avez une base solide de compétences en programmation. C'est particulièrement important si vous êtes un·e jeune développeur·se ou si vous êtes en reconversion professionnelle.

    \item \textbf{Incluez votre expérience professionnelle :} Même si vous n'avez pas beaucoup d'expérience en tant que développeur·se, incluez toute expérience professionnelle que vous avez. Cela peut inclure des stages, des emplois à temps partiel, ou des emplois non liés à la programmation où vous avez acquis des compétences transférables.

    Même si vous n'avez pas beaucoup d'expérience en tant que développeur·se, toute expérience professionnelle peut être utile. Elle montre que vous avez une expérience de travail et que vous avez développé des compétences transférables qui peuvent être utiles dans un environnement de travail.

    \item \textbf{Soignez la présentation :} Assurez-vous que votre CV est bien présenté, facile à lire et sans fautes d'orthographe ou de grammaire. Utilisez des puces pour rendre votre CV plus lisible et n'hésitez pas à utiliser un modèle de CV professionnel.

    Un CV bien présenté peut faire une grande différence. Il montre que vous êtes professionnel·le et que vous prenez au sérieux votre recherche d'emploi. Un CV bien présenté est également plus facile à lire, ce qui peut aider les employeur·se·s à trouver rapidement les informations qu'ils·elles recherchent.
\end{itemize}

En suivant ces conseils, vous pouvez rédiger un CV de développeur qui met en valeur vos compétences, votre expérience et votre passion pour le développement, et qui vous aide à vous démarquer des autres candidats.

\section{Exemple de CV de développeur}
Pour illustrer ces conseils, voici un exemple de CV pour un jeune développeur nommé Alex, qui vient de terminer une école et est à la recherche de son premier emploi en tant que développeur web.

\begin{framed}
    \addtolength{\leftskip}{5mm} % Ajoute une marge à gauche
    \addtolength{\rightskip}{5mm} % Ajoute une marge à droite
    \addtolength{\topskip}{5mm} % Ajoute une marge en haut

    \fontsize{9pt}{10pt}\selectfont
    \begin{tabular}{l c}
        \textbf{Nom}       & Alex Dupont           \\
        \textbf{Email}     & alex.dupont@email.com \\
        \textbf{Téléphone} & +33 6 12 34 56 78     \\
        \textbf{Portfolio} & www.alexdupont.com    \\
        \textbf{GitHub}    & github.com/alexdupont \\
    \end{tabular}

    \begin{center}
        \large   Développeur Web Junior
    \end{center}

    \textbf{Résumé professionnel :} Développeur web junior passionné, maîtrisant HTML, CSS, JavaScript et React. Diplômé de l'école de programmation XYZ, avec une expérience en tant que stagiaire en technologie. Recherche un poste de développeur web junior pour approfondir mes compétences et contribuer à des projets stimulants.

    \textbf{Compétences techniques :} HTML, CSS, JavaScript, React, Git, Node.js, MongoDB

    \textbf{Projets personnels :}
    \begin{itemize}
        \item \textbf{TodoApp :} Une application de liste de tâches interactive créée avec React. \textit{github.com/alexdupont/todoapp}
        \item \textbf{PortfolioArtist :} Un site web de portfolio créé pour un artiste local en utilisant HTML et CSS. \textit{github.com/alexdupont/portfolioartist}
        \item \textbf{PuzzleGame :} Un jeu de puzzle créé en utilisant JavaScript. \textit{github.com/alexdupont/puzzlegame}
    \end{itemize}
    \vspace{10pt}
    \textbf{Formation :} École de programmation XYZ, Programme de développement web, 2022

    \textbf{Expérience professionnelle :}
    \begin{itemize}
        \item \textbf{2022-Actuellement, Alternance en technologie, Entreprise ABC :}
              \subitem Mise en place des tests des applications
              \subitem Résolution des problèmes techniques.
        \item \textbf{2021-2022, Serveur, Restaurant DEF :}
              \subitem Développement de compétences en service à la clientèle et en gestion du temps.
    \end{itemize}
    \vspace{15pt}
\end{framed}


\subsection{En-tête}

Alex commence par son nom, son titre professionnel (dans ce cas, "Développeur·se Web Junior"), et ses coordonnées, y compris son adresse e-mail, son numéro de téléphone, et le lien vers son portfolio en ligne et son profil GitHub. Il est important d'inclure ces informations pour que les employeur·se·s potentiel·le·s puissent facilement le·la contacter et voir son travail.

\subsection{Résumé professionnel}

Ensuite, Alex inclut un bref résumé professionnel. Ce résumé donne un aperçu de ses compétences, de son expérience et de ses objectifs de carrière. Il est important d'inclure un résumé professionnel car il donne aux employeur·se·s une idée rapide de qui vous êtes et de ce que vous pouvez apporter à leur entreprise.

\subsection{Compétences techniques}

Alex liste ensuite ses compétences techniques. Il inclut les langages de programmation qu'il·elle maîtrise (par exemple, HTML, CSS, JavaScript, et React), ainsi que d'autres compétences techniques pertinentes (par exemple, Git, Node.js, et MongoDB). Il est crucial de mettre en avant vos compétences techniques car elles sont souvent la première chose que les employeur·se·s recherchent chez un·e développeur·se.

\subsection{Projets personnels}

Alex inclut ensuite une section sur ses projets personnels. Pour chaque projet, il·elle donne le nom du projet, une brève description de ce qu'il fait, les technologies qu'il·elle a utilisées, et un lien vers le code source sur GitHub. Inclure des projets personnels est une excellente façon de montrer vos compétences pratiques, surtout si vous n'avez pas beaucoup d'expérience professionnelle. Cela montre également votre passion pour le développement et votre capacité à mener à bien des projets en dehors d'un environnement de travail structuré.

\subsection{Formation}

Alex liste ensuite sa formation. Il·elle inclut le nom de l'école de programmation qu'il·elle a fréquentée, le titre du programme ou du cours qu'il·elle a suivi, et les dates de fréquentation. Même si la formation formelle n'est pas toujours nécessaire pour devenir développeur·se, elle peut aider à établir votre crédibilité et à montrer que vous avez une base solide de compétences en programmation.

\subsection{Expérience professionnelle}

Enfin, Alex inclut une section sur son expérience professionnelle. Comme il·elle s'agit de son premier emploi en tant que développeur·se, il·elle n'a pas beaucoup d'expérience professionnelle dans le développement. Cependant, il·elle inclut son expérience en tant que stagiaire dans une entreprise de technologie, où il·elle a aidé à tester des applications et à résoudre des problèmes techniques. Il·elle inclut également son expérience en tant que serveur·se dans un restaurant, où il·elle a développé des compétences transférables comme le service à la clientèle et la gestion du temps. Même si cette expérience n'est pas directement liée au développement, elle montre qu'Alex a une expérience de travail et des compétences qui peuvent être utiles dans un environnement de travail.

En suivant ces conseils et cet exemple, vous pouvez rédiger un CV de développeur·se impressionnant qui met en valeur vos compétences, votre expérience et votre passion pour le développement.

