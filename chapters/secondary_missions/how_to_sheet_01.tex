\chapter{Fiche pratique : Contribuer à un projet Open Source}

Un guide étape par étape pour trouver un projet Open Source, comprendre son code et faire votre première contribution.

Contribuer à un projet Open Source est une excellente façon d'acquérir de l'expérience en développement. Non seulement cela vous permet de pratiquer vos compétences en codage, mais cela vous expose également à de nouvelles technologies et méthodologies, vous permet de travailler avec d'autres développeur·euse·s, et peut même vous donner une certaine visibilité dans la communauté du développement. Voici un guide étape par étape pour vous aider à faire votre première contribution à un projet Open Source.

\section{Trouver un projet Open Source}

La première étape pour contribuer à un projet Open Source est de trouver un projet qui vous intéresse. Il existe des milliers de projets Open Source disponibles, couvrant une grande variété de technologies et de domaines d'application. Voici quelques conseils pour trouver un projet qui correspond à vos intérêts et à votre niveau de compétence :

\begin{itemize}
    \item \textbf{Explorez GitHub :} GitHub est la plateforme la plus populaire pour l'hébergement de projets Open Source. Vous pouvez rechercher des projets par technologie, par langue de programmation, par niveau de difficulté, ou même par nombre de contributeur·rice·s.
    \item \textbf{Choisissez quelque chose qui vous passionne :} Vous serez plus motivé·e à contribuer à un projet qui vous intéresse vraiment. Que vous soyez passionné·e par le développement web, l'intelligence artificielle, les jeux vidéo, ou tout autre domaine, il y a probablement un projet Open Source qui correspond à vos intérêts.
    \item \textbf{Commencez petit :} Si vous êtes nouveau·elle dans la contribution à l'Open Source, il peut être préférable de commencer par un petit projet ou un projet qui a une communauté accueillante pour les débutant·e·s. De nombreux projets utilisent des tags comme "good first issue" ou "beginner-friendly" pour indiquer les tâches qui sont appropriées pour les nouveaux·elles contributeur·rice·s.
\end{itemize}

Par exemple, si vous êtes intéressé·e par le développement web et que vous voulez contribuer à un projet qui utilise React, vous pouvez rechercher "React" sur GitHub et parcourir les résultats pour trouver un projet qui vous plaît.

\section{Comprendre le code du projet}

Une fois que vous avez trouvé un projet qui vous intéresse, la prochaine étape est de comprendre son code. Cela peut être un défi, surtout si le projet est grand ou complexe, mais ne vous inquiétez pas - vous n'avez pas besoin de comprendre tout le code tout de suite. Voici quelques conseils pour commencer :

\begin{itemize}
    \item \textbf{Lisez la documentation :} La plupart des projets Open Source ont une documentation qui explique comment le projet fonctionne, comment le configurer et l'utiliser, et comment contribuer. La lecture de cette documentation peut vous donner une bonne idée de la structure et du fonctionnement du projet.
    \item \textbf{Explorez le code :} Prenez le temps de parcourir le code du projet. Essayez de comprendre comment les différentes parties du code interagissent entre elles. Si vous ne comprenez pas quelque chose, n'hésitez pas à faire des recherches ou à demander de l'aide.
    \item \textbf{Exécutez le projet :} Si possible, essayez d'exécuter le projet sur votre propre machine. Cela peut vous aider à comprendre comment le projet fonctionne en pratique et peut également vous permettre de tester et de déboguer le code.
\end{itemize}

Par exemple, si vous avez choisi de contribuer à un projet de développement web, vous pouvez commencer par lire la documentation du projet, explorer le code HTML, CSS et JavaScript, et essayer d'exécuter le site web sur votre propre machine.

\section{Faire votre première contribution}

Maintenant que vous avez trouvé un projet et que vous avez commencé à comprendre son code, vous êtes prêt·e à faire votre première contribution. Voici comment vous pouvez procéder :

\begin{itemize}
    \item \textbf{Choisissez une tâche :} Commencez par choisir une tâche à accomplir. Cela pourrait être une fonctionnalité à ajouter, un bug à corriger, ou même une erreur de frappe dans la documentation. De nombreux projets Open Source ont une liste d'issues ou de tâches qui ont besoin d'être résolues, donc c'est un bon endroit pour commencer.
    \item \textbf{Fork et clone le projet :} Pour commencer à travailler sur le projet, vous devrez d'abord le "forker" sur GitHub, ce qui crée une copie du projet dans votre propre compte GitHub. Ensuite, vous pouvez "cloner" le projet sur votre machine locale, ce qui vous permet de travailler sur le code.
    \item \textbf{Créez une branche :} Il est de bonne pratique de créer une nouvelle branche pour chaque tâche sur laquelle vous travaillez. Cela vous permet de travailler sur différentes tâches en parallèle sans interférer les unes avec les autres.
    \item \textbf{Faites vos modifications :} Une fois que vous avez une branche pour votre tâche, vous pouvez commencer à faire vos modifications. Assurez-vous de suivre les conventions de codage du projet et de tester votre code pour vous assurer qu'il fonctionne correctement.
    \item \textbf{Soumettez une pull request :} Lorsque vous avez terminé vos modifications et que vous êtes prêt·e à les soumettre au projet, vous pouvez créer une "pull request" sur GitHub. Cela informe les mainteneur·euse·s du projet de votre contribution et leur permet de l'examiner et de l'accepter.
\end{itemize}

Par exemple, si vous avez choisi de contribuer à un projet de développement web, vous pouvez choisir de corriger un bug dans le code JavaScript, forker et cloner le projet, créer une nouvelle branche pour votre tâche, faire vos modifications, et ensuite soumettre une pull request.

Contribuer à un projet Open Source peut sembler intimidant au début, mais avec un peu de pratique, cela devient un processus enrichissant et gratifiant. N'oubliez pas que l'objectif est d'apprendre et de gagner de l'expérience, alors n'ayez pas peur de faire des erreurs et de demander de l'aide. Bonne chance dans votre quête pour devenir un·e contributeur·rice Open Source !

\section{Exemple de contribution à un projet Open Source}

Pour illustrer le processus de contribution à un projet Open Source, prenons l'exemple d'un projet fictif appelé "OpenWebApp". Supposons que vous ayez trouvé ce projet sur GitHub, que vous soyez intéressé·e par le développement web et que vous souhaitiez contribuer à ce projet.

\subsection{Choisir une tâche}

En explorant la liste des issues du projet "OpenWebApp" sur GitHub, vous trouvez une issue intitulée "Correction d'un bug dans la fonction de recherche". Vous décidez de travailler sur cette tâche.

\subsection{Forker et cloner le projet}

Vous commencez par forker le projet "OpenWebApp" sur GitHub. Cela crée une copie du projet dans votre propre compte GitHub. Ensuite, vous clonez le projet sur votre machine locale en utilisant la commande git clone.

\subsection{Créer une branche}

Avant de commencer à travailler sur la tâche, vous créez une nouvelle branche en utilisant la commande git branch. Vous nommez la branche "correction-bug-recherche", ce qui donne la commande suivante : git branch correction-bug-recherche.

\subsection{Faire vos modifications}

Vous ouvrez le code du projet dans votre éditeur de code préféré et commencez à travailler sur la correction du bug. Après avoir fait vos modifications, vous testez le code pour vous assurer que le bug a été corrigé et que tout le reste fonctionne correctement.

\subsection{Soumettre une pull request}

Une fois que vous êtes satisfait·e de vos modifications, vous les commitez à votre branche en utilisant la commande git commit. Ensuite, vous poussez vos modifications sur GitHub en utilisant la commande git push. Enfin, vous allez sur GitHub et créez une pull request pour soumettre vos modifications au projet "OpenWebApp".

En suivant ces étapes, vous avez réussi à faire votre première contribution à un projet Open Source. Félicitations !

