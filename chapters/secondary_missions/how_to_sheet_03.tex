\chapter{Fiche pratique : Création et maintenance d'un portfolio}
Un guide pratique pour créer un portfolio de développement, y ajouter des projets et le maintenir à jour.

Un portfolio de développement est une collection de vos travaux qui démontre vos compétences, vos réalisations et votre expérience en tant que développeur·euse. Il peut être un outil précieux lorsque vous postulez à des emplois ou à des stages, car il permet aux employeur·euse·s potentiel·le·s de voir concrètement ce que vous êtes capable de faire. Voici un guide pratique pour vous aider à créer et à maintenir votre propre portfolio de développement.

\section{Création d'un portfolio}

La première étape pour créer un portfolio est de choisir quels projets vous voulez y inclure. Voici quelques conseils pour choisir vos projets :

\begin{itemize}
    \item \textbf{Choisissez des projets qui démontrent vos compétences :} Votre portfolio doit montrer ce que vous êtes capable de faire. Choisissez des projets qui démontrent vos compétences en codage, votre capacité à résoudre des problèmes, et votre créativité.
    \item \textbf{Incluez une variété de projets :} Si possible, essayez d'inclure une variété de projets qui montrent différentes compétences et technologies. Cela peut aider à montrer votre polyvalence et votre capacité à apprendre de nouvelles choses.
    \item \textbf{Incluez des projets que vous êtes fier·ère :} Votre portfolio est une représentation de vous-même en tant que développeur·euse. Assurez-vous d'inclure des projets que vous êtes fier·ère et qui représentent bien votre travail.
\end{itemize}

Une fois que vous avez choisi vos projets, vous pouvez commencer à créer votre portfolio. Il existe de nombreux outils et plateformes que vous pouvez utiliser pour créer votre portfolio, comme GitHub Pages, Jekyll, ou WordPress. Choisissez un outil qui correspond à vos compétences et à vos besoins.

\section{Ajouter des projets à votre portfolio}

Pour chaque projet que vous incluez dans votre portfolio, vous devriez fournir quelques informations clés :

\begin{itemize}
    \item \textbf{Une description du projet :} Qu'est-ce que le projet fait ? Quel problème résout-il ? Quelles technologies avez-vous utilisées ?
    \item \textbf{Des captures d'écran ou des démos :} Les visuels peuvent aider les gens à comprendre rapidement ce que fait votre projet et à voir la qualité de votre travail. Si possible, incluez des captures d'écran ou des démos de votre projet.
    \item \textbf{Un lien vers le code source :} Cela permet aux autres de voir votre code et de comprendre comment vous avez construit le projet. Si possible, incluez un lien vers le code source du projet sur GitHub ou une autre plateforme de partage de code.
\end{itemize}

\section{Maintenance de votre portfolio}

Une fois que vous avez créé votre portfolio, il est important de le maintenir à jour. Voici quelques conseils pour maintenir votre portfolio :

\begin{itemize}
    \item \textbf{Ajoutez de nouveaux projets :} À mesure que vous créez de  nouveaux projets, ajoutez-les à votre portfolio. Cela montre que vous continuez à apprendre et à développer vos compétences.
    \item \textbf{Mettez à jour les informations sur les projets existants :} Si vous apportez des modifications significatives à un projet existant, mettez à jour les informations sur ce projet dans votre portfolio.
    \item \textbf{Retirez les projets obsolètes :} Si un projet n'est plus représentatif de vos compétences actuelles, il peut être préférable de le retirer de votre portfolio.
\end{itemize}

Un portfolio de développement est un outil puissant pour montrer vos compétences et vos réalisations en tant que développeur·euse. En suivant ces conseils, vous pouvez créer un portfolio impressionnant qui vous aidera à vous démarquer et à réussir dans votre carrière de développeur·euse.

\section{Exemple de création et de maintenance d'un portfolio}

Pour illustrer le processus de création et de maintenance d'un portfolio, prenons l'exemple d'un·e développeur·euse fictif·ive appelé·e Alex. Alex est un·e développeur·euse web qui maîtrise HTML, CSS, JavaScript et React.

\subsection{Création d'un portfolio}

Alex décide de créer un portfolio pour montrer ses compétences et ses réalisations en tant que développeur·euse web. Il·elle choisit d'inclure trois de ses meilleurs projets : une application de liste de tâches qu'il·elle a créée en utilisant React, un site web de portfolio qu'il·elle a créé pour un·e artiste local·e en utilisant HTML et CSS, et un jeu de puzzle qu'il·elle a créé en utilisant JavaScript.

Alex décide d'utiliser GitHub Pages pour créer son portfolio car il·elle est déjà familier·ère avec GitHub et il·elle aime la simplicité de GitHub Pages. Il·elle crée une nouvelle page GitHub, ajoute des informations sur lui·elle-même et ses compétences, et ajoute des sections pour chacun de ses projets.

\subsection{Ajouter des projets à son portfolio}

Pour chaque projet, Alex ajoute une description du projet, des captures d'écran, et un lien vers le code source sur GitHub. Par exemple, pour son application de liste de tâches, il·elle explique qu'il·elle a utilisé React pour créer une application interactive qui permet aux utilisateur·rice·s de gérer leurs tâches. Il·elle inclut des captures d'écran de l'application et un lien vers le code source sur GitHub.

\subsection{Maintenance de son portfolio}

Au fil du temps, Alex continue à apprendre de nouvelles technologies et à créer de nouveaux projets. Chaque fois qu'il·elle termine un nouveau projet qu'il·elle est fier·ère de montrer, il·elle l'ajoute à son portfolio. Il·elle met également à jour les informations sur ses projets existants lorsque nécessaire, et retire les projets qui ne sont plus représentatifs de ses compétences actuelles.

En suivant ces étapes, Alex a réussi à créer un portfolio impressionnant qui montre ses compétences et ses réalisations en tant que développeur·euse web. Il·elle utilise régulièrement son portfolio lorsqu'il·elle postule à des emplois ou à des stages, et il·elle a reçu de nombreux commentaires positifs sur la qualité de son travail.

