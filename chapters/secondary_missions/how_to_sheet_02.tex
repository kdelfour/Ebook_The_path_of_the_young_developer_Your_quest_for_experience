\chapter{Fiche pratique : Création d'un projet personnel}

Les projets personnels sont un excellent moyen d'acquérir de l'expérience en développement. Ils vous permettent de pratiquer vos compétences en codage, d'explorer de nouvelles technologies, et de créer quelque chose de tangible que vous pouvez montrer aux autres. De plus, ils peuvent être une excellente addition à votre portfolio lorsque vous postulez à des emplois ou à des stages. Voici un guide étape par étape pour vous aider à créer votre propre projet personnel.

\section{Choisir un projet}

La première étape pour créer un projet personnel est de choisir un projet qui vous intéresse. Il est important de choisir un projet qui vous passionne, car vous serez plus motivé·e à travailler dessus et à le terminer. Voici quelques conseils pour choisir un projet :

\begin{itemize}
    \item \textbf{Choisissez quelque chose qui vous passionne :} Que vous soyez intéressé·e par le développement web, les jeux vidéo, l'intelligence artificielle, ou tout autre domaine, choisissez un projet qui correspond à vos intérêts.
    \item \textbf{Choisissez un projet adapté à votre niveau de compétence :} Si vous êtes un·e débutant·e, il peut être préférable de choisir un projet relativement simple. Si vous êtes plus expérimenté·e, vous pouvez choisir un projet plus complexe qui vous mettra au défi.
    \item \textbf{Choisissez un projet qui vous permettra d'apprendre quelque chose de nouveau :} Les projets personnels sont une excellente occasion d'apprendre de nouvelles technologies ou méthodologies. Choisissez un projet qui vous permettra d'élargir vos compétences et vos connaissances.
\end{itemize}

Par exemple, si vous êtes intéressé·e par le développement web et que vous voulez apprendre React, vous pouvez choisir de créer une application web simple en utilisant React.

\section{Planifier le projet}

Une fois que vous avez choisi un projet, la prochaine étape est de le planifier. Une bonne planification peut vous aider à organiser votre travail, à définir des objectifs clairs et à éviter les problèmes potentiels. Voici quelques conseils pour planifier votre projet :

\begin{itemize}
    \item \textbf{Définissez les fonctionnalités de base :} Quelles sont les fonctionnalités minimales que votre projet doit avoir ? C'est ce qu'on appelle souvent le "Minimum Viable Product" ou MVP. Définir votre MVP vous donne un objectif clair à atteindre.
    \item \textbf{Créez une timeline :} Combien de temps prévoyez-vous de passer sur ce projet ? Créez une timeline avec des étapes ou des milestones pour vous aider à suivre votre progression.
    \item \textbf{Prévoyez du temps pour l'apprentissage :} Si votre projet implique l'apprentissage de nouvelles technologies ou méthodologies, assurez-vous de prévoir du temps pour cela.
\end{itemize}

Par exemple, si vous avez choisi de créer une application web en utilisant React, votre MVP pourrait être une application de liste de tâches simple avec des fonctionnalités de base comme l'ajout de nouvelles tâches, la suppression de tâches existantes et la mise à jour de tâches. Vous pourriez prévoir de passer une semaine à apprendre les bases de React, puis deux semaines à développer l'application.

\section{Développer le projet}

Maintenant que vous avez un plan, il est temps de commencer à développer votre projet. Voici quelques conseils pour vous aider dans cette étape :

\begin{itemize}
    \item \textbf{Commencez par les fonctionnalités de base :} Concentrez-vous d'abord sur le développement de votre MVP. Cela vous donnera un sentiment d'accomplissement et vous permettra d'avoir quelque chose à montrer rapidement.
    \item \textbf{Testez votre code :} Assurez-vous de tester votre code régulièrement pour vous assurer qu'il fonctionne comme prévu. Cela peut vous aider à identifier et à corriger les bugs plus tôt.
    \item \textbf{Utilisez le contrôle de version :} Utilisez un système de contrôle de version comme Git pour suivre vos modifications et vous permettre de revenir à une version précédente de votre code si nécessaire.
\end{itemize}

Par exemple, si vous développez une application de liste de tâches en utilisant React, vous pouvez commencer par développer la fonctionnalité d'ajout de nouvelles tâches, puis tester cette fonctionnalité pour vous assurer qu'elle fonctionne correctement.

\section{Finaliser le projet}

Une fois que vous avez développé toutes les fonctionnalités de votre projet et que vous êtes satisfait·e de votre travail, il est temps de finaliser le projet. Voici quelques étapes que vous pourriez vouloir suivre :

\begin{itemize}
    \item \textbf{Testez le projet :} Assurez-vous de tester l'ensemble du projet pour vous assurer qu'il fonctionne correctement. Cela pourrait impliquer des tests unitaires, des tests d'intégration, et des tests manuels.
    \item \textbf{Documentez le projet :} Créez une documentation pour votre projet qui explique ce qu'il fait, comment l'utiliser, et comment contribuer. Cela peut être très utile si vous voulez partager votre projet avec d'autres ou si vous voulez y revenir plus tard.
    \item \textbf{Publiez le projet :} Enfin, publiez votre projet sur une plateforme comme GitHub. Cela vous permet de partager votre travail avec d'autres, de recevoir des commentaires, et même de collaborer avec d'autres développeur·euse·s.
\end{itemize}

Par exemple, une fois que vous avez terminé votre application de liste de tâches en React, vous pouvez la tester, créer une documentation, et la publier sur GitHub.

Créer un projet personnel peut être un défi, mais c'est aussi une opportunité incroyable d'apprendre, de pratiquer vos compétences, et de créer quelque chose de tangible. Alors, qu'attendez-vous ? Commencez à planifier votre propre projet personnel aujourd'hui !

\section{Exemple de création d'un projet personnel}

Pour illustrer le processus de création d'un projet personnel, prenons l'exemple d'un projet fictif appelé "MyTodoApp". Supposons que vous ayez décidé de créer une application de liste de tâches en utilisant React.

\subsection{Choisir un projet}

Vous êtes intéressé·e par le développement web et vous voulez apprendre React. Vous décidez donc de créer une application de liste de tâches simple en utilisant React. Cette application permettra aux utilisateur·rice·s d'ajouter de nouvelles tâches, de marquer les tâches comme terminées et de supprimer les tâches.

\subsection{Planifier le projet}

Vous commencez par définir les fonctionnalités de base de votre application : ajouter de nouvelles tâches, marquer les tâches comme terminées et supprimer les tâches. Vous décidez de passer une semaine à apprendre les bases de React, puis deux semaines à développer l'application.

\subsection{Développer le projet}

Vous commencez par apprendre les bases de React en suivant des tutoriels en ligne. Ensuite, vous commencez à développer votre application. Vous commencez par la fonctionnalité d'ajout de nouvelles tâches, puis vous développez la fonctionnalité de marquage des tâches comme terminées, et enfin la fonctionnalité de suppression des tâches. Vous testez chaque fonctionnalité à mesure que vous la développez pour vous assurer qu'elle fonctionne correctement.

\subsection{Finaliser le projet}

Une fois que vous avez développé toutes les fonctionnalités de votre application et que vous êtes satisfait·e de votre travail, vous testez l'ensemble de l'application pour vous assurer qu'elle fonctionne correctement. Ensuite, vous créez une documentation pour votre application qui explique ce qu'elle fait et comment l'utiliser. Enfin, vous publiez votre application sur GitHub pour partager votre travail avec d'autres.

En suivant ces étapes, vous avez réussi à créer votre propre projet personnel. Félicitations !